\documentclass{article}
\usepackage[margin=0.5in,landscape]{geometry}
\usepackage{stix2}
\usepackage{multicol}
\usepackage{xcolor}

\pagestyle{empty}
\usepackage{tikz}
\usepackage{pgfplots}
\pgfplotsset{compat=1.18}

\usepackage{siunitx}

\usepackage{verbatim}

\newenvironment{basic}{\comment}{\endcomment}


\usepackage{listings}
\lstset{language=[LaTeX]TeX,
keepspaces=true,
basicstyle=\color{blue!75!black}\tt,% 
% macros
morekeywords={addplot,draw,pgfplotsset,tikzexternalize,
tikzsetexternalprefix,tikzsetnextfilename,usetikzlibrary},
keywordstyle=\color{blue!50!green},
columns=flexible, % prevents listings from adding extra space between letters
commentstyle=\color{red!45!black},
aboveskip=0pt,
belowskip=\medskipamount,
moredelim=[is][\itshape\color{gray}]{"}{"},
literate=<{$\langle$}{1} >{$\rangle$}{1},
xleftmargin=10pt
}
\lstMakeShortInline"

\usepackage{relsize}
\newcommand{\code}[1]{{\relscale{1.1}\tt\color{blue!75!black}#1}}

\lstnewenvironment{latex}[1][]{\lstset{,#1}}{}

\usepackage{savetrees}
\titlespacing{\section}
  {0pt}% left
  {4pt}% before
  {0pt}% after
\setcounter{secnumdepth}{-1}
\parindent=0in


% expressions should end at column 46
\begin{document}
\centerline{\large\bf PGFPLOTS Cheat Sheet for Calculus-type graphs}
\begin{multicols}{3}
\begin{basic}
\section{Basic environments}
Setup 
\begin{latex}
\usepackage{pgfplots}
\pgfplotsset{compat=newest}
\end{latex}

Basic command
\begin{latex}
\begin{tikzpicture}["<options>"]
\begin{axis}["<options>"]
\end{axis}
\end{tikzpicture}
\end{latex}
\end{basic}

\section{Addplot}
\begin{latex}
\addplot coordinates {"<coordinate list>"};
\addplot table ["<column selection">]{"<table>"};
\addplot {"<math expression>"};
\addplot[variable = t] ( {"<math>"}, {"<math>"} );
\end{latex}

Coordinate list example:
\begin{latex}
{(1,2) (10,12) (25,30)}
\end{latex}

Inline table example
\begin{latex}
{
1 1
2 4
3 9
}; %  must be by itself on last line
\end{latex}

Math expression example
\begin{latex}
\addplot{e^{-x^2}};
\end{latex}

\section{Coordinates}
Cartesian coordinates
\begin{latex}
(length1,length2)
\end{latex}

Polar coordinates
\begin{latex}
(angle:length)
\end{latex}

Coordinate math expression examples:\\
"(pi/2, {sin(pi/2)})" \quad or \quad "(1e6, 2e7)"

\begin{basic}
\section{Tikz commands}
Straight line
\begin{latex}
\draw "<coord>" -- "<coord>";
\end{latex}

Curve
\begin{latex}[xleftmargin=0pt]
\draw[out="<angle>",in="<angle>"] "<coord>" to "<coord>";
\end{latex}

Relative positions 
\begin{latex}
\draw "<coord>" -- ++ "<coord>";
\end{latex}

Circle
\begin{latex}
\draw "<coord>" circle ("<dim>");
\end{latex}

\begingroup \lstset{literate=<{<}{1} >{>}{1}}
Arrows Example\\
\begin{minipage}{1.8in}
\begin{latex}[xleftmargin=0in,belowskip=0pt]
\draw[<->] (0,0) -- (1,0);
\end{latex}
\end{minipage}
\qquad \tikz[<->] \draw (0,0) -- (1,0);\\
(Can add "<" and/or ">" and/or "|" to "-")
\endgroup
\medskip

Nodes:
\begin{latex}
\draw "<coord>" node["options"] {"<contents>"};
\end{latex}

Node options:
\begin{latex}[deletekeywords={draw}]
left|right|above|below|above left|"<etc">
pos = "<percent>"
draw % draws boundary of node
circle % changes boundary to circle
inner sep = "<dim>"
text width = "<dim>"
\end{latex}

Font example
\begin{latex}
\begin{tikzpicture}[font={\small\bfseries}]
\end{latex}
\end{basic}

\section{Options for \code{axis}}
Box, $x$-axis, $y$-axis, etc
\begin{latex}
axis lines=box|left|middle|center|right|none
axis x line=box|top|middle|center|bottom|none
axis y line=box|left|middle|center|right|none
\end{latex}

Labels
\begin{latex}
xlabel = "<text>", ylabel = "<text>"
\end{latex}

Title
\begin{latex}
title = "<text>"
\end{latex}

Ticks
\begin{latex}[belowskip=0pt]
ticks = minor|major|both|none
xtick = \empty|"<number list>"
ytick = \empty|"<number list>"
xtick distance = "<dim>" % dist between ticks
ytick distance = "<dim>"
minor tick num = "<num>" % ticks between ticks
xticklabels="<text list>"
yticklabels="<text list>"
\end{latex}
(WARNING: "xticklabel" is an unrelated command)\medskip
\columnbreak

Grids
\begin{latex}
grid=minor|major|both|none 
\end{latex}

Ticklabel scale format
\begin{latex}
tick scale binop="<binary operator">
\end{latex}
%\vspace{-\medskipamount}

\section{Axis external sizing options}
Size styles
\begin{itemize}
\renewcommand{\labelitemi}{}
\item "normalsize" :  \hfill \qty{8.4}{cm} $\times$  \qty{7.3}{cm} 
          =  \qty{3.3}{in} $\times$ \qty{2.9}{in}
\item "small" : \hfill \qty{6.50}{cm} $\times$ \qty{5.60}{cm} 
         = \qty{2.52}{in} $\times$ \qty{2.20}in
\item "footnotesize" :  \hfill \qty{5}{cm} $\times$ \qty{4.31}{cm} 
       = \qty{1.97}{in} $\times$ \qty{1.70}{in}
\item "tiny" : \hfill \qty{4}{cm} $\times$ \qty{3.45}{cm} 
         = \qty{1.57}{in} $\times$ \qty{1.36}{in}
\end{itemize}
\medskip

Setting width and height
\begin{latex}
\begin{axis}[width="<dim>", height = "<dim>"]
\end{latex}

Scaling whole image
\begin{latex}
\begin{tikzpicture}[scale=1.25] % visual mag
\begin{axis}[scale=1.25] % logical mag
\end{latex}

Changing external aspect ratio
\begin{latex}
\begin{axis}[y post scale=2] % change height
\begin{axis}[height=9cm]  % change height
\end{latex}

Clipping
\begin{latex}
clip = true|false
clip mode = global|individual
restrict y to domain = "<min>":"<max>"
\end{latex}

\section{Axis internal sizing options}
Setting internal limits
\begin{latex}
xmin = "<number>", xmax = "<number>", 
ymin = "<number>", ymin = "<number>"
\end{latex}

Changing internal aspect ratio
\begin{latex}
\begin{axis}[axis equal]
\begin{axis}[unit vector ratio = "<num>" "<num>"]
\end{latex}
% The latter makes the vertical and horizontal scales different: $1$ in
% the horizontal direction is $20$ times longer visually than $1$ in the
% vertical.  

\section{Options for Addplot (mostly from Tikz)}
Line thickness:
\begin{latex}
ultra thin|very thin|thin|semithick|thick
   |very thick|ultra thick|linewidth="<dim>"
\end{latex}
% The first 7 of these produce lines of thickness 0.1pt, 0.2pt, 0.4pt,
% 0.6pt, 0.8pt, 1.2pt, 1.6pt, respectively.\medskip

Line style:
\begin{latex}
solid|dashed|dotted|dash dot|dash dot dot
     |densely dashed|loosely dashed|"<etc>"
\end{latex}

Color Example:
\begin{latex}[belowskip=0pt]
red!50!black,ultra thick
\end{latex}
\vspace{-5pt}
\tikz \draw[red!50!black,ultra thick] (0,0) -- (6,0);\medskip

Line doubling example
\begin{latex}[belowskip=0pt]
double=white,thick
\end{latex}
\vspace{-5pt}
\tikz\draw[double=white,thick] (0,0) -- (6,0);\medskip
\columnbreak

Line shape
\begin{latex}
sharp|smooth
\end{latex}

Where functions are evaluated
\begin{latex}
domain = "<min>":"<max>"
samples = "<whole number>" % num of values used
samples at = {"<numbers>"} % manual override
\end{latex}

\section{Marks}
\begin{latex}
only marks % don't connect points
marks = none 
mark = *|+|x|o|star|square|oplus|diamond|"<etc>"
mark size = "<dim>"
mark indices = {"<index list>"} % which points
mark options = {"<style declarations>"}
\end{latex}

Style declarations example
\begin{latex}
mark options = {scale=2,thick,fill=white}
\end{latex}

\section{Math options}
Setting trig functions in plot commands to use radians
\begin{latex}
trig format plots = rad
\end{latex}

Defining constant and function example:
\begin{latex}[literate=<{<}{1} >{>}{1}]
declare function = { a = 5; % <- semicolon!
f(\x) = (\x-a)^2; % <- semicolon!
} 
\end{latex}

% things I left out:
% opacity 


\section{Defining styles}
Storing a style in a name
\begin{latex}[xleftmargin=0pt]
\pgfplotsset{duckplot/.style={thick,blue,smooth}}
\addplot[duckplot]{x^2};
\end{latex}

Setting a style for whole document
\begin{latex}
\pgfplotsset{
  every axis/.style = {axis lines = middle},
  cycle list name = color list,
  every axis plot/.style = {thick,smooth}
  }
\end{latex}

\section{Externalize}
\begin{latex}
\usetikzlibrary{external}
\tikzexternalize
\tikzsetexternalprefix{tikzfigs/}
\tikzsetnextfilename{3p5_example_1}
\end{latex}

Command line
\begin{latex}[belowskip=0pt]
pdflatex -shell-escape mainfile
\end{latex}
(Note: "mainfile" CANNOT have spaces in name)
\end{multicols}


\end{document}
%%% Local Variables:
%%% mode: latex
%%% TeX-master: t
%%% End: