\section{Breaking up and externalizing}
\label{section_externalization}
\subsection{Breaking up}
Using "pgfplots" with lots of plots that have lots of points can slow
a document down.  The oldest, most basic approach to improving a slow
document is to break your document up into different parts and then,
when you're editing and making lots of changes, just typeset one part
at a time.  The standard way to do this is to use "\includeonly" and
"\include", something like this
\begin{latex}
"\includeonly{section1}" % only section 1 will be typeset

\begin{document}
"\include{section1}"
\include{section2}
\include{section3}
% etc.
\end{document}
\end{latex}
When you're ready to produce the whole document, just put a comment
symbol "%"
in front of "\includeonly" and the whole document will be typeset.  I
have done that for this document and it's helped a lot: this
\emph{whole} document takes \qty{6.1}{s} but this section alone takes
\qty{0.7}{s}.

\subsection{Externalizing}
Odds are you can skip this section unless you are making a document
that is hundreds of pages and/or has hundreds of figures.  As
mentioned in Section~\ref{section_pushing_limit}, speed can start
to be an issue in using "pgfplots" if you have a document that has
lots of figures and they have a large number of "samples".  In a
program like Matlab you might not hesitate at all in running a plot
with 300, or even 1000 points.  But TeX, after all, was written during
1978--1990, uses fixed point arithmetic and was designed to calculate
vertical and horizontal placement of boxes: it's not really designed
to be fast at 1000 calculations of floating point numbers.

\hypersetup{linkcolor=black} % turn off coloring links

As an example, the current version of this document is
\pageref{lastpage} pages, has \ref{lastpic}\ figures, and takes
\qty{6.1}{s} on my current MacBook Air~(M4, 2026). At that rate it
would take about \qty{2}{min} for a 1000 page book, which is probably
too long to be convenient. But the PGF package includes a library
(sub-package) written by the author of "pgfplots" that helps out.

When one uses the "externalize" library this is what happens: When
Tikz finds a "tikzpicture" environment it checks to see if it is new
in that document, and/or if it has changed since the last time the
document was typeset.  If so, it creates a temporary document that
contains just that picture, typesets it, and saves the PDF that has
just that picture.  Then it inserts that PDF into the main
document. On subsequent typesetting runs, it checks again, and
assuming nothing has changed in that picture, it simply inserts the
PDF into the main document.  Thus, on subsequent runs, the amount of
time needed is roughly just the same as a simple call to
"\includegraphics".  This can represent a huge speed increase.  

Here's an example of how I use it
\begin{latex}
% in preamble
\usetikzlibrary{external}
\tikzexternalize
\tikzsetexternalprefix{tikzfigs/}

% in main document
\tikzsetnextfilename{3p5_example_1}
\begin{tikzpicture}
...
\end{tikzpicture}
\end{latex}
Note that to activate "externalize" you need to do two things: 
load "external", \emph{and} issue the "\tikzexternalize"
command.  This way you can keep the package loaded, but comment out
"\tikzexternalize" to turn off externalizing, and just run things like
normal.  

It's not necessary to use "\tikzsetexternalprefix" but it's a good
idea, because it puts the files created by "external" in a separate
folder. It's not necessary to use "\tikzsetnextfilename", without it
the library will automatically create a file name for the external
image, but this automatic feature is fragile in the sense that if you
insert a new picture (or delete an old one) then the automatic file
names are thrown off (they are named with a number like
``"picture_0"''), and so I use "\tikzsetnextfilename" (actually I
created my own shorter command name which is just an alias for this command).  

After doing the above you need to typeset the file with a command that
has extra permissions; for Linux and MacOS systems one can use the
following:
\begin{latex}
pdflatex -shell-escape mainfile
\end{latex}
(Warning: "mainfile" CANNOT have spaces in its name).  On subsequent
runs you can use your usual typeset command.

For further details the reader should refer to the Tikz/PGF
user manual.

%%% Local Variables:
%%% mode: latex
%%% TeX-master: "../pgfplots_tutorial"
%%% End:
